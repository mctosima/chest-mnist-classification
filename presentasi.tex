% Presentasi Eksperimen Deep Learning
% Martin - 12345 - Biomedis ITERA

\documentclass[aspectratio=169]{beamer}
\usetheme{Madrid}
\usecolortheme{default}

% Packages
\usepackage[utf8]{inputenc}
\usepackage[indonesian]{babel}
\usepackage{graphicx}
\usepackage{amsmath}
\usepackage{listings}
\usepackage{xcolor}

% Info
\title{Eksperimen Klasifikasi Citra Medis}
\subtitle{Perbandingan Arsitektur Deep Learning}
\author{Martin (12345)}
\institute{Program Studi Biomedis\\Institut Teknologi Sumatera (ITERA)}
\date{\today}

% Custom colors
\definecolor{codegreen}{rgb}{0,0.6,0}
\definecolor{codegray}{rgb}{0.5,0.5,0.5}
\definecolor{codepurple}{rgb}{0.58,0,0.82}
\definecolor{backcolour}{rgb}{0.95,0.95,0.92}

\begin{document}

% Slide 1: Title
\begin{frame}
\titlepage
\end{frame}

% Slide 2: Latar Belakang & Tujuan
\begin{frame}{Latar Belakang \& Tujuan}
\begin{block}{Latar Belakang}
\begin{itemize}
    \item Klasifikasi citra medis memerlukan model yang akurat dan efisien
    \item Berbagai arsitektur deep learning menawarkan trade-off antara akurasi dan kompleksitas
    \item Perlu dilakukan perbandingan untuk menemukan model terbaik
\end{itemize}
\end{block}

\vspace{0.5cm}

\begin{block}{Tujuan Penelitian}
\begin{itemize}
    \item Membandingkan performa 4 arsitektur deep learning berbeda
    \item Menganalisis efisiensi komputasi dari setiap model
    \item Menentukan model terbaik untuk klasifikasi citra medis
\end{itemize}
\end{block}
\end{frame}

% Slide 3: Arsitektur Model
\begin{frame}{Arsitektur Model yang Diuji}
\begin{columns}[T]
\begin{column}{0.5\textwidth}
\begin{block}{1. Simple CNN}
\begin{itemize}
    \item CNN sederhana dengan 2 layer konvolusi
    \item \textbf{Parameters:} $\sim$50K
    \item \textbf{Kelebihan:} Cepat, ringan
    \item \textbf{Kekurangan:} Akurasi terbatas
\end{itemize}
\end{block}

\begin{block}{2. ResNet-18}
\begin{itemize}
    \item 18 layer dengan skip connections
    \item \textbf{Parameters:} $\sim$11M
    \item \textbf{Kelebihan:} Training stabil
    \item \textbf{Kekurangan:} Lebih berat
\end{itemize}
\end{block}
\end{column}

\begin{column}{0.5\textwidth}
\begin{block}{3. ResNet-34}
\begin{itemize}
    \item 34 layer dengan residual blocks
    \item \textbf{Parameters:} $\sim$21M
    \item \textbf{Kelebihan:} Akurasi tinggi
    \item \textbf{Kekurangan:} Komputasi berat
\end{itemize}
\end{block}

\begin{block}{4. ShuffleNet V1}
\begin{itemize}
    \item Channel shuffle + depthwise conv
    \item \textbf{Parameters:} $\sim$1-2M
    \item \textbf{Kelebihan:} Sangat efisien
    \item \textbf{Kekurangan:} Implementasi kompleks
\end{itemize}
\end{block}
\end{column}
\end{columns}
\end{frame}

% Slide 4: Metodologi
\begin{frame}{Metodologi Eksperimen}
\begin{block}{Dataset}
\begin{itemize}
    \item \textbf{Dataset:} ChestMNIST (Medical MNIST)
    \item \textbf{Jumlah Kelas:} 2 (klasifikasi biner)
    \item \textbf{Ukuran Gambar:} 224 $\times$ 224 piksel
    \item \textbf{Channel:} 1 (grayscale)
\end{itemize}
\end{block}

\begin{block}{Hyperparameter Training}
\begin{itemize}
    \item \textbf{Optimizer:} Adam
    \item \textbf{Learning Rate:} 0.00001
    \item \textbf{Batch Size:} 16
    \item \textbf{Epochs:} 16
    \item \textbf{Loss Function:} BCEWithLogitsLoss
\end{itemize}
\end{block}
\end{frame}

% Slide 5: Implementasi
\begin{frame}[fragile]{Implementasi Sistem}
\begin{block}{Struktur Kode}
\begin{itemize}
    \item \texttt{model.py} - Simple CNN (baseline)
    \item \texttt{model\_resnet.py} - ResNet-18 \& ResNet-34
    \item \texttt{model\_shufflenet.py} - ShuffleNet V1
    \item \texttt{train.py} - Script training dengan model selection
    \item \texttt{datareader.py} - Data loading \& preprocessing
\end{itemize}
\end{block}

\begin{block}{Pemilihan Model}
\begin{lstlisting}[language=Python, basicstyle=\tiny\ttfamily, 
    backgroundcolor=\color{backcolour},
    keywordstyle=\color{blue},
    stringstyle=\color{codepurple}]
# Pilih model dengan mengubah variable
MODEL_TYPE = 'shufflenet'  # Options: 
# 'simple_cnn', 'resnet18', 'resnet34', 
# 'shufflenet', 'shufflenet_small'
\end{lstlisting}
\end{block}

\begin{alertblock}{Fleksibilitas}
Sistem dirancang modular: \textbf{mudah menambah model baru} atau \textbf{mengganti hyperparameter} tanpa mengubah struktur kode utama.
\end{alertblock}
\end{frame}

% Backup slide: Kesimpulan (optional)
\begin{frame}{Terima Kasih}
\begin{center}
\Huge Terima Kasih!

\vspace{1cm}

\Large \textbf{Martin (12345)}\\
\normalsize Program Studi Biomedis\\
Institut Teknologi Sumatera

\vspace{1cm}

\normalsize
\textit{``Deep Learning untuk Kesehatan yang Lebih Baik''}
\end{center}
\end{frame}

\end{document}
